% Exam question booklets must only have the exam and qn options enabled
\documentclass[exam,soln]{yqteach}

%% Add in any additional settings needed here


%%%%%%%%%%%%%%%%%%%%%%%%%%%%%%%%%%%%%%%%%%%%%

% set the appropriate details for the module here
% Set details about the module here
\setinstitution{National University of Singapore}
\setdepartment{Department of Computer Science, School of Computing}
\setmodulecode{CS9999}
\setmoduletitle{Teaching Document Templates with \LaTeX}
\setfullacademicyear{2000/2001}
\setshortacademicyear{00/01}
\setsemester{Semester 2}
\setcopyrightyear{2023}

% Write the details of the exam here
\setdocumenttitle{Final Assessment}
\setduedate{1 January 2000}
\settimeallowed{1 hour}
%%%%%%%%%%%%%%%%%%%%%%%%%%%%%%%%%%%%

\begin{document}

\maketitle

% Use the \examsection command to start an exam section. The first argument is the section title, and the second argument is the total marks of this section
\examsection{Multiple Choice Questions}{2}


% Use the \q command to start a question. The argument denotes the number of marks this question is worth.
% For example, \q{3} gives a new question worth 3 marks.
\q{1} \answer \ansopt{B}. Users should be provided with the \emph{least} privilege necessary to do their job.

%%% Don't touch the following

\endofdocument

\end{document}
