% Exam question booklets must only have the exam and qn options enabled
\documentclass[exam,qn]{yqteach}

%% Add in any additional settings needed here
\usepackage{minted}
\usemintedstyle{bw}
%%%%%%%%%%%%%%%%%%%%%%%%%%%%%%%%%%%%%%%%%%%%%

% set the appropriate details for the module here
% Set details about the module here
\setinstitution{National University of Singapore}
\setdepartment{Department of Computer Science, School of Computing}
\setmodulecode{CS9999}
\setmoduletitle{Teaching Document Templates with \LaTeX}
\setfullacademicyear{2000/2001}
\setshortacademicyear{00/01}
\setsemester{Semester 2}
\setcopyrightyear{2023}

% Write the details of the exam here
\setdocumenttitle{Final Assessment}
\setduedate{1 January 2000}
\settimeallowed{1 hour}
\settotalscore{100}
\settotalweight{20}
%%%%%%%%%%%%%%%%%%%%%%%%%%%%%%%%%%%%

\begin{document}

\maketitle

\begin{instructions}
    % page count, question count, total score, total weightage and plagiarism/cheating notice are already pre-loaded
    \item This is an \textbf{open-book}, \textbf{open-internet} assessment. You are allowed to refer to materials and access any resources (online or otherwise), except:
    \begin{itemize}
        \item You are \textbf{NOT allowed} to communicate with anyone. You must force-close your email client, messaging applications like Telegram, WeChat, etc before starting the quiz.
        \item You are \textbf{NOT allowed} to post on any forums, such as Reddit, StackOverflow etc. as this constitutes communicating with others.
        \item You are \textbf{NOT allowed} to use ChatGPT or similar tools using Large Language Models (LLMs) to generate answers.
    \end{itemize}
    \item You are \textbf{allowed} to use any other application of your choice, such as using the Python interpreter, MS Excel, other IDEs etc.
    \item \textbf{Mobile phones and smartwatches should be turned off.}
\end{instructions}

% Use the \examsection command to start an exam section. The first argument is the section title, and the second argument is the total marks of this section
\examsection{Multiple Choice Questions}{2}


% Use the \q command to start a question. The argument denotes the number of marks this question is worth.
% For example, \q{3} gives a new question worth 3 marks.
% You may also add an additional topic like \q[Algebra]{4} to give Question 1 (Algebra) [4 marks].
\q{1} Which of the following statements is \textbf{not} a fundamental concept in security?
% Use the letteroptions environment for MCQ letter options
\begin{letteroptions}
    \item Assume Breach: organizations should take active steps to mitigate, monitor, detect and deal with threats to their applications and systems.
    \item Maximum Functionality: privileged users (like administrators) should be given the maximum amount of access to features of their applications to deal with threats swiftly.
    \item Defence-in-Depth: Having multiple layers of security so that threats are mitigated even if they successfully gets around one security measure.
    \item Attack Surface Reduction: creating smaller systems that have a lower attack surface so that there are fewer vectors that an attacker may take to exploit the application.
\end{letteroptions}

\q{1} Which of the following is \textbf{not} an example of Defence-in-Depth?
\begin{letteroptions}
    \item Requiring that clients authenticate themselves with an application using a username and password pair, in addition to proving the ownership of a mobile device they registered previously.
    \item Performing input validation on the client-side on top of input validation on the server side.
    \item Using SameSite (with the strict option) cookies in conjunction with adding anti-CSRF tokens onto forms to prevent Cross-Site Request Forgeries (CSRFs).
    \item Performing password hashing on the client-side instead of performing password hashing on the server side.
\end{letteroptions}

\examsection{Short Answer Questions}{3}
There are two short answer questions below. Keep your answers simple and concise.

\q{3} Suppose we have a statement performing an SQL query from un-sanitized user input \texttt{someinput}:

\begin{minted}{javascript}
const p = req.body.someinput
const q = await db.query(`SELECT * FROM table WHERE a = '${p}';`)
\end{minted}
Clearly, if the user enters something like \mintinline{text}{'} (just a single quotation mark), the query would become\\
\mintinline{text}{SELECT * FROM table WHERE a = ''';} which is malformed, and thus the query would throw an exception.

However, if prepared statements (parameterized queries) are used, then the user input is sanitized and encoded so that the query is no longer malformed.

Charlie says, "as long as we use prepared statements, the query \texttt{db.query(... some prepared statement ...)} will never throw an exception and exception handling is thus not required."

% Use the \sq command to enter a subquestion
\sq{2} Tell Charlie he is wrong by giving him a scenario in which attempting to perform a well-formed SQL query to a database may still yield an exception.

\sq{1} How can Charlie handle exceptions?

%%% Don't touch the following

\endofdocument

\end{document}
