% Exam question booklets must only have the exam and qn options enabled
\documentclass[ass,qn]{yqteach}

%% Add in any additional settings needed here

%%%%%%%%%%%%%%%%%%%%%%%%%%%%%%%%%%%%%%%%%%%%%

% set the appropriate details for the module here
% Set details about the module here
\setinstitution{National University of Singapore}
\setdepartment{Department of Computer Science, School of Computing}
\setmodulecode{CS9999}
\setmoduletitle{Teaching Document Templates with \LaTeX}
\setfullacademicyear{2000/2001}
\setshortacademicyear{00/01}
\setsemester{Semester 2}
\setcopyrightyear{2023}

% Write the details of the exam here
\setdocumenttitle{Assignment 1}
\setdocumentsubtitle{Design}
\setduedate{1 January 2000, 11.59pm}
\settotalscore{100}
\settotalweight{20}
%%%%%%%%%%%%%%%%%%%%%%%%%%%%%%%%%%%%

\begin{document}

\maketitle


% Use the \examsection command to start an exam section. The first argument is the section title, and the second argument is the total marks of this section
\examsection{Multiple Choice Questions}{2}


% Use the \q command to start a question. The argument denotes the number of marks this question is worth.
% For example, \q{3} gives a new question worth 3 marks.
% You may also add an additional topic like \q[Algebra]{4} to give Question 1 (Algebra) [4 marks].
\q[Security Fundamentals]{1} Which of the following statements is \textbf{not} a fundamental concept in security?

% Use the letteroptions environment for MCQ letter options
\begin{letteroptions}
    \item Assume Breach: organizations should take active steps to mitigate, monitor, detect and deal with threats to their applications and systems.
    \item Maximum Functionality: privileged users (like administrators) should be given the maximum amount of access to features of their applications to deal with threats swiftly.
    \item Defence-in-Depth: Having multiple layers of security so that threats are mitigated even if they successfully gets around one security measure.
    \item Attack Surface Reduction: creating smaller systems that have a lower attack surface so that there are fewer vectors that an attacker may take to exploit the application.
\end{letteroptions}

%%% Don't touch the following

\endofdocument

\end{document}
